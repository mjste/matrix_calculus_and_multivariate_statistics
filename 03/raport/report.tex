\documentclass[a4paper]{article}
\usepackage[utf8]{inputenc}
\usepackage[T1]{fontenc}
\usepackage[polish]{babel}
\usepackage{graphicx}
\usepackage{listings}
\usepackage{float}
\usepackage{amsmath}
\usepackage{geometry}
 \geometry{
 a4paper,
 total={170mm,257mm},
 left=20mm,
 top=20mm,
 }

\graphicspath{ {./images/} }


\begin{document}


\begin{titlepage}
    \begin{center}
        \vspace*{1cm}

        \Huge
        \textbf{Rachunek Macierzowy i Statystyka Wielowymiarowa}

        \Large
        \vspace{0.5cm}
        Raport z zadania 3. - Implementacja norm macierzowych, wspołczynników uwarunkowania oras SVD.

        \vspace{1.5cm}

        \textbf{Wojciech Jasiński, Michał Stefanik}

        \vfill


        \vspace{0.8cm}

        \includegraphics[width=0.4\textwidth]{agh_logo.jpg}

        Wydział Informatyki\\
        Akademia Górniczo Hutnicza\\
        Kraków\\
        \today

    \end{center}
\end{titlepage}

\tableofcontents
\newpage




\section{Wstęp}


Należało zaimplementować:
\begin{itemize}
    \item Normy macierzowe: $||A||_1$, $||A||_2$, $||A||_p$, $||A||_{\infty}$ 
    \item Współczynniki uwarunkowania: $cond_1(A)$, $cond_2(A)$, $cond_p(A)$, $cond_{\infty}(A)$
    \item Rozkład SVD macierzy
\end{itemize}



\section{Dane techniczne}
Używamy \texttt{np.ndarray} z biblioteki \texttt{numpy} do reprezentacji macierzy. Wszystkie funkcje przyjmują i zwracają macierze w tej postaci.


Do testów używamy macierzy o wymiarach 3x3:

\[
    M =
\begin{bmatrix}
    4 & 9 & 2\\
    3 & 5 & 7\\
    8 & 1 & 6
\end{bmatrix}
\]



\section{Normy macierzowe}

\subsection{Norma $||A||_1$}

\subsubsection{Algorytm}
Wzór na tę normę to 
$||A||_1 = max_{j-1,...,n}\sum_{i=1,...,n}|a_{ij}|$. W prostszych słowach,
jest to maksymalna suma wartości bezwzględnych w kolumnach macierzy.

\subsubsection{Fragment kodu}

\begin{lstlisting}[language=python]
def matrix_norm_1(A: np.ndarray):
    return np.max(np.sum(np.abs(A), axis=0))
\end{lstlisting}

\subsubsection{Wartość dla macierzy testowej}

Dla macierzy testowej $M$ wartość normy $||M||_1$ wynosi 15.

\subsection{Norma $||A||_2$}

\subsubsection{Algorytm}
Wzór na tę normę to
$||A||_2 = |{\lambda_{1}}|$. W prostszych słowach, to największa
co do modułu wartość własna macierzy.

\subsubsection{Fragment kodu}

\begin{lstlisting}[language=python]
def matrix_norm_2(A: np.ndarray):
    return np.max(np.abs(np.linalg.eigvals(A)))
\end{lstlisting}

\subsection{Wartość dla macierzy testowej}

Dla macierzy testowej $M$ wartość normy $||M||_2$ wynosi 15.

\subsection{Norma $||A||_p$}

\subsection{Norma $||A||_{\infty}$}

\subsubsection{Algorytm}
Wzór na tę normę to
$||A||_{\infty} = max_{i=1,...,n}\sum_{j=1,...,n}|a_{ij}|$. W prostszych słowach,
jest to maksymalna suma wartości bezwzględnych w wierszach macierzy.

\subsubsection{Fragment kodu}

\begin{lstlisting}[language=python]
def matrix_norm_inf(A: np.ndarray):
    return np.max(np.sum(np.abs(A), axis=1))
\end{lstlisting}

\subsection{Wartość dla macierzy testowej}

Dla macierzy testowej $M$ wartość normy $||M||_{\infty}$ wynosi 15.


\section{Współczynniki uwarunkowania}

\subsection{Współczynnik uwarunkowania $cond_1(A)$}

\subsection{Współczynnik uwarunkowania $cond_2(A)$}

\subsection{Współczynnik uwarunkowania $cond_p(A)$}

\subsection{Współczynnik uwarunkowania $cond_{\infty}(A)$}

\section{Rozkład SVD macierzy}

\subsection{Algorytm}

Macierz \(A\) można zdekomponować na trzy macierze: \(U\), \(S\) i \(V\), takie że:
\[
    A = U \cdot S \cdot V^T
\]

gdzie:
\begin{itemize}
    \item \(U\) - macierz ortonormalna
    \item \(S\) - macierz diagonalna
    \item \(V\) - macierz ortonormalna
\end{itemize}

Macierz \(S\) zawiera wartości osobliwe macierzy \(A\), a macierze \(U\) i \(V\)
zawierają wektory własne macierzy odpowiednio $A A^T $ oraz $A^T  A$.

Macierz V znajdujemy jako macierz wektorów własnych macierzy \(A^T A\).
Następnie sortujemy wartości własne malejąco i tworzymy macierz V z wektorów własnych
odpowiadających wartościom własnym w tej kolejności. Wartości własne dla tej macierzy
pierwiastkujemy i tworzymy z nich macierz S.

Następnie obliczamy macierz U jako \(U = A \cdot V \cdot S^{-1}\).

\subsection{Fragment kodu}

\begin{lstlisting}[language=python]
def SVD(A: np.ndarray):
    left_shape = A.shape[0]
    right_shape = A.shape[1]
    eigenvalues, V = np.linalg.eigh(np.dot(A.T, A))

    idx = np.argsort(eigenvalues)[::-1]
    eigenvalues = eigenvalues[idx]
    V = V[:, idx]

    singular_values = np.sqrt(eigenvalues)
    right_singular_vectors = V

    left_singular_vectors = np.dot(A, right_singular_vectors)

    with np.errstate(divide="ignore"):
        left_singular_vectors /= singular_values

    left_singular_vectors = left_singular_vectors[:, :left_shape]
    Sigma = np.diag(singular_values)[:left_shape, :right_shape]

    return left_singular_vectors, Sigma, right_singular_vectors.T

\end{lstlisting}

\subsection{Wartość dla macierzy testowej}

Dla macierzy testowej $M$ otrzymujemy:

\[
    U =
\begin{bmatrix}
    -0.45 & -0.89\\
    -0.98 & -.45
\end{bmatrix}
\]

\[
    S =
\begin{bmatrix}
    3 & 0 & 0\\
    0 & 2 & 0
\end{bmatrix}
\]

\[
    V^T =
\begin{bmatrix}
    0.7 & -0.3 & -0.5\\
    0 & -0.8 & 0.4\\
    -0.67 & -.33  & 0.67
\end{bmatrix}
\]

\section{Wnioski}

W przypadku SVD zastosowana metoda pozwala nie szukac dwukrotnie wartości własnych, przez co
może być szybsza.


\end{document}